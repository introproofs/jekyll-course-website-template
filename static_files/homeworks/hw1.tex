\documentclass[12pt,english]{amsart}
\usepackage{amsmath,amssymb,amsthm}
\usepackage{geometry}
\geometry{verbose,tmargin=.8in,bmargin=1in,lmargin=1in,rmargin=1in}
\usepackage[all]{xy}
\usepackage{bussproofs}

\theoremstyle{plain}
\newtheorem{theorem}{Theorem}[section]
\theoremstyle{definition}
\newtheorem{definition}[theorem]{Definition}
\theoremstyle{plain}
\newtheorem{lemma}[theorem]{Lemma}
\theoremstyle{plain}
\newtheorem{example}[theorem]{Example}
\theoremstyle{remark}
\newtheorem{proposition}[theorem]{Proposition}

\newcounter{problems}
\theoremstyle{definition}
\newtheorem{problem}[problems]{Problem}
\newtheorem{bproblem}[problems]{ Problem*}

\theoremstyle{remark}
\newtheorem*{solution}{Solution}


% redefining the \LaTeX command for implication operators so that they look neater. 

\renewcommand{\implies}{\Rightarrow}
\newcommand{\biimplies}{\ensuremath{\Leftrightarrow}}

% Introduction
\title{Math 301: Introduction to Proofs\\
Homework 1}

\author[]{Sina Hazratpour}

\begin{document}
\maketitle



%% problem 1

\begin{problem}
	Prove that, for all real numbers $x$ and $y$, if $x$ is irrational, then $x+y$ and $x-y$ are not both rational.
\end{problem}


%% Write your solution in below between \begin{solution} and \end{solution}. 

%\begin{solution}
%	
%\end{solution}	


%% problem 2

\begin{problem}
Suppose $a$ and $n$ are non-zero natural number. 
Show that the proposition 
\begin{center}
    ``If $a^n - 1 $ is prime, then $n$ is prime''
\end{center}
is not necessarily true. 
What other condition about $a$ can we add to the conclusion so that the proposition becomes valid? Then give a proof of the modified proposition.
\end{problem}



% problem 3
\begin{problem}
Let $P$, $Q$ and $R$ be propositions. 
\begin{enumerate}  
    \item Construct a proof of proposition
	$$P\vee (Q \wedge R) \Rightarrow (P\vee Q)\wedge (P\vee R)$$
	using natural deduction. 
	\item Draw the associated tree of your deduction in the first part.
\end{enumerate}
\end{problem}

% \begin{solution}
	
% \end{solution}	 


%% problem 4

\begin{problem}
In this exercise we’ll learn about Peirce’s law, a curiosity of \emph{classical} logic.
Let $P$, $Q$ and $R$ be propositions.  \leavevmode 
\begin{enumerate}
    \item With the help of 
    the axiom of double negation  (which says for any proposition $A$, the proposition $A \biimplies \neg \neg A$ is a tautology) 
    construct a proof of proposition $$ ((P \implies Q) \implies P) \implies P $$ 
    in the style of natural deduction. 
    \item Verify the result of part 1 by drawing a truth table.  
\end{enumerate}
\end{problem}


%% problem 5
\begin{problem}
	On an island inhabited by knights and knaves, where the former always tell the truth and the latter always lie, you meet three individuals: Alice, Bob, and Eve. Alice says that Bob is a knight. Bob say that Alice is a knight but Eve is a knave. Eve says that both Alice and Bob are knights. Determine who is a knight and who is a knave by constructing a truth table.
\end{problem}

% \begin{solution}
	
% \end{solution}	


	 

%% problem 6 - bonus problem
\begin{bproblem}
Show that there are irrational numbers $x$ and $y$ 
such that 
$x^y$ is a rational number. 
\end{bproblem}


%\begin{solution}
%	
%\end{solution}	


\end{document}
